\documentclass{article}
\usepackage[utf8]{inputenc}
\usepackage[T1]{fontenc}
\usepackage[francais]{babel}
\usepackage{minted}
\usepackage{uqac}

%<----------->Meta-Data<----------->

\title{Premier Travail}
\author{Tristan MOLIN\\Florian FICHANT\\Martin LOCQUEVILLE}
\codep{}
\discipline{8INF914 - Visualisation Analytique}
\projet{Etude des algorithmes de dessin d'arbres \\ Travail de session \\ Rapport}
\date{\today}

%<--------------------------------->

\begin{document}

\maketitle

\tableofcontents

\newpage

\mainmatter

\section{Introduction}

L'objectif de notre projet est de parvenir à implémenter en \emph{Python}, au sein de l'outil d'étude de graphe \emph{Tulip}, un algorithme de dessin d'arbre, le plus optimal possible. Pour ce faire, nous avons recueilli et analysé les travaux de recherche dans quatre articles.

Ces articles avaient également comme problématique l'optimisation de dessin de graphe. Chacun d'eux cherchant à optimiser l'implémentation d'un algorithme de dessin d'arbre en prenant en compte les travaux de leurs prédécesseurs (Nous avons pris soin de noter les références de ces articles en fin de rapport).

Suite à l'étude de ces articles, nous allons premièrement implémenter les algorithmes résultant des travaux des auteurs de nos quatre articles de référence. C'est après cela que nous allons écrire de manière optimisée et implémenter notre solution algorithmique en \emph{Python}, en utilisant la bibliothèque \emph{Tulip}.



\newpage
\section{Étude préliminaire des articles}
\subsection{subsection name}
\subsubsection{subsubsection name}

\newpage
\section{section3 name}






\newpage
\medskip

\begin{thebibliography}{10}

\bibitem{article79}
C. Wetherell and A. Shannon. Tidy Drawings of trees. \textit{IEEE transactions on Software Engineering} SE-5, 5 (septembre 1979) p514-520.

\bibitem{article81}
E. Reingold and J. Tilford. Tidier drawings of trees. \textit{IEEE Transactions on Software Engineering} SE-7, 2 (mars 1981) p223-228.

\bibitem{article90}
J. Walker II. A node-positioning algorithm for general trees. \textit{Software – Practice and Experience} SE-20, 2 (1990) p685–705.

\bibitem{article02}
C. Buchheim, M. Jünger and S. Leipert. Improving Walker’s algorithm to run in linear time. Technical Report zaik2002-431, ZAIK, Universität zu Köln, (2002) p344-352.

\end{thebibliography}

\end{document}
